\documentclass[a4paper]{article}
% generated by Docutils <https://docutils.sourceforge.io/>
\usepackage{cmap} % fix search and cut-and-paste in Acrobat
\usepackage{ifthen}
\usepackage[T1]{fontenc}
\usepackage[utf8]{inputenc}
\usepackage{alltt}

%%% Custom LaTeX preamble
% PDF Standard Fonts
\usepackage{mathptmx} % Times
\usepackage[scaled=.90]{helvet}
\usepackage{courier}

%%% User specified packages and stylesheets
\usepackage{alectryon}
\usepackage{pygments}

%%% Fallback definitions for Docutils-specific commands

% hyperlinks:
\ifthenelse{\isundefined{\hypersetup}}{
  \usepackage[colorlinks=true,linkcolor=blue,urlcolor=blue]{hyperref}
  \usepackage{bookmark}
  \urlstyle{same} % normal text font (alternatives: tt, rm, sf)
}{}
\hypersetup{
  pdftitle={LaTeX and HTML dialects},
}

%%% Body
\begin{document}
\title{LaTeX and HTML dialects%
  \label{latex-and-html-dialects}}
\author{}
\date{}
\maketitle

\let\oldalltt\alltt
\def\alltt{\oldalltt\scriptsize}

This simple file demos LaTeX and HTML dialect configuration:

\begin{quote}
\begin{alltt}
alectryon -{}-html-dialect=html4 -o dialects.4.html dialects.rst
  # HTML4; produces ‘dialects.4.html’
alectryon -{}-html-dialect=html5 -o dialects.5.html dialects.rst
  # HTML5; produces ‘dialects.5.html’

alectryon -{}-latex-dialect=pdflatex -o dialects.tex dialects.rst
  # LaTeX; produces ‘dialects.tex’
alectryon -{}-latex-dialect=xelatex -o dialects.xe.tex dialects.rst
  # XeLaTeX; produces ‘dialects.xe.tex’
alectryon -{}-latex-dialect=lualatex -o dialects.lua.tex dialects.rst
  # LuaLaTeX; produces ‘dialects.lua.tex’
\end{alltt}
\end{quote}

\begin{alectryon}
  % Generator: Alectryon
  \sep
  \begin{sentence}
    \begin{input}
      \PY{k+kn}{Goal}~\PY{k+kt}{True}\PY{o}{.}
    \end{input}
    \sep
    \begin{output}
      \begin{goals}
        \begin{goal}
          \begin{hyps}\end{hyps}
          \sep
          \infrule{}
          \sep
          \begin{conclusion}
            \PY{k+kt}{True}
          \end{conclusion}
        \end{goal}
      \end{goals}
    \end{output}
  \end{sentence}
  \sep
  \begin{sentence}
    \begin{input}
      ~~\PY{n+nb+bp}{exact}~\PY{n}{I}\PY{o}{.}\nl
    \end{input}
  \end{sentence}
  \sep
  \begin{sentence}
    \begin{input}
      ~~\PY{k+kn}{Show~Proof}\PY{o}{.}
    \end{input}
    \sep
    \begin{output}
      \begin{messages}
        \begin{message}
          \PY{n}{I}
        \end{message}
      \end{messages}
    \end{output}
  \end{sentence}
\end{alectryon}

\end{document}
