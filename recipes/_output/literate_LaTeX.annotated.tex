\documentclass[10pt]{beamer}

\usetheme{default}

\title{Writing Alectryon documents in LaTeX}

\usepackage{alectryon}
\usepackage{pygments}

\begin{document}
\maketitle

\begin{frame}[fragile]{Compiling this example}
  In LaTeX mode alectryon takes a LaTeX document as input, and produces a LaTeX
  document as output, with code blocks annotated:

  \begin{verbatim}
$ alectryon literate_LaTeX.tex
  # LaTeX → LaTeX,
    produces ‘literate_LaTeX.annotated.tex’
  \end{verbatim}
\end{frame}

\begin{frame}[fragile]{Showing input-output examples}
  In LaTeX Coq fragments are written as \verb|\begin{alectryon}{coq}{}|:

\begin{alectryon}
  \begin{\al{sentence}}
    \begin{\al{input}}
      \PY{k+kn}{Goal}~\PY{k}{exists}~\PY{n+nv}{x}\PY{o}{,}~\PY{n}{x}~\PY{o}{*}~\PY{n}{x}~\PY{o}{=}~\PY{l+m+mi}{49}~\PY{o}{/\PYZbs{}}~\PY{n}{x}~\PY{o}{\PYZlt{}}~\PY{l+m+mi}{10}\PY{o}{.}
    \end{\al{input}}
    \Al{sep}
    \begin{\al{output}}
      \begin{\al{goals}}
        \begin{\al{goal}}
          \begin{\al{hyps}}\end{\al{hyps}}
          \Al{sep}
          \Al{infrule}{}
          \Al{sep}
          \begin{\al{conclusion}}
            \PY{k}{exists}~\PY{n+nv}{x}~\PY{o}{:}~\PY{n}{nat}\PY{o}{,}~\PY{n}{x}~\PY{o}{*}~\PY{n}{x}~\PY{o}{=}~\PY{l+m+mi}{49}~\PY{o}{/\PYZbs{}}~\PY{n}{x}~\PY{o}{\PYZlt{}}~\PY{l+m+mi}{10}
          \end{\al{conclusion}}
        \end{\al{goal}}
      \end{\al{goals}}
    \end{\al{output}}
  \end{\al{sentence}}
  \Al{sep}
  \begin{\al{sentence}}
    \begin{\al{input}}
      ~~\PY{k}{let~rec}~\PY{n+nv}{t}~\PY{n+nv}{n}~\PY{o}{:=}\Al{nl}
      ~~~~\PY{o}{(}\PY{k}{exists}~\PY{n+nv}{n}\PY{o}{;}~\PY{n+nb}{split}\PY{o}{;}~\PY{o}{[}\PY{n+nb+bp}{reflexivity}~\PY{o}{|])}~\PY{o}{||}~\PY{n}{t}~\PY{o}{(}\PY{n}{S}~\PY{n}{n}\PY{o}{)}~\PY{k}{in}\Al{nl}
      ~~\PY{n}{t}~\PY{l+m+mi}{0}\PY{o}{.}
    \end{\al{input}}
    \Al{sep}
    \begin{\al{output}}
      \begin{\al{goals}}
        \begin{\al{goal}}
          \begin{\al{hyps}}\end{\al{hyps}}
          \Al{sep}
          \Al{infrule}{}
          \Al{sep}
          \begin{\al{conclusion}}
            \PY{l+m+mi}{7}~\PY{o}{\PYZlt{}}~\PY{l+m+mi}{10}
          \end{\al{conclusion}}
        \end{\al{goal}}
      \end{\al{goals}}
    \end{\al{output}}
  \end{\al{sentence}}
  \Al{sep}
  \begin{\al{sentence}}
    \begin{\al{input}}
      ~~\PY{n+nb}{unfold}~\PY{n}{lt}\PY{o}{.}
    \end{\al{input}}
    \Al{sep}
    \begin{\al{output}}
      \begin{\al{goals}}
        \begin{\al{goal}}
          \begin{\al{hyps}}\end{\al{hyps}}
          \Al{sep}
          \Al{infrule}{}
          \Al{sep}
          \begin{\al{conclusion}}
            \PY{l+m+mi}{8}~\PY{o}{\PYZlt{}=}~\PY{l+m+mi}{10}
          \end{\al{conclusion}}
        \end{\al{goal}}
      \end{\al{goals}}
    \end{\al{output}}
  \end{\al{sentence}}
  \Al{sep}
  \begin{\al{sentence}}
    \begin{\al{input}}
      ~~\PY{k+kp}{repeat}~\PY{n+nb}{constructor}\PY{o}{.}\Al{nl}
    \end{\al{input}}
  \end{\al{sentence}}
  \Al{sep}
  \begin{\al{sentence}}
    \begin{\al{input}}
      \PY{k+kn}{Qed}\PY{o}{.}
    \end{\al{input}}
  \end{\al{sentence}}
\end{alectryon}
\end{frame}

\begin{frame}[fragile]{Customizing output}
  IO comments work as usual, but you can also put IO flags in the second argument of the \verb|{alectryon}| environment to specify the same flags that you would put on the first line of a \verb|.. coq::| directive.

  Note also that special characters in \verb|{alectryon}| environments don't need to be escaped, since Alectryon is used as a preprocessor.



\begin{alectryon}
  \begin{\al{sentence}}
    \begin{\al{output}}
      \begin{\al{messages}}
        \begin{\al{message}}
          \PY{n}{The}~\PY{n}{reference}~\PY{n}{z}~\PY{n}{was}~\PY{n}{not}~\PY{n}{found}~\PY{k}{in}~\PY{n}{the}~\PY{n}{current}\Al{nl}
          \PY{n}{environment}\PY{o}{.}
        \end{\al{message}}
      \end{\al{messages}}
    \end{\al{output}}
  \end{\al{sentence}}
\end{alectryon}
\end{frame}

\begin{frame}{Help wanted!}
  We'd love to add support for converting between Coq and LaTeX, just like we have support for converting between Coq and reST.  Please send patches!
\end{frame}
\end{document}
