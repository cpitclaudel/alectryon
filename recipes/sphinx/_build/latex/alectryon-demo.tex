%% Generated by Sphinx.
\def\sphinxdocclass{report}
\documentclass[letterpaper,10pt,english]{sphinxmanual}
\ifdefined\pdfpxdimen
   \let\sphinxpxdimen\pdfpxdimen\else\newdimen\sphinxpxdimen
\fi \sphinxpxdimen=.75bp\relax
\ifdefined\pdfimageresolution
    \pdfimageresolution= \numexpr \dimexpr1in\relax/\sphinxpxdimen\relax
\fi
%% let collapsible pdf bookmarks panel have high depth per default
\PassOptionsToPackage{bookmarksdepth=5}{hyperref}

\PassOptionsToPackage{booktabs}{sphinx}
\PassOptionsToPackage{colorrows}{sphinx}

\PassOptionsToPackage{warn}{textcomp}
\usepackage[utf8]{inputenc}
\ifdefined\DeclareUnicodeCharacter
% support both utf8 and utf8x syntaxes
  \ifdefined\DeclareUnicodeCharacterAsOptional
    \def\sphinxDUC#1{\DeclareUnicodeCharacter{"#1}}
  \else
    \let\sphinxDUC\DeclareUnicodeCharacter
  \fi
  \sphinxDUC{00A0}{\nobreakspace}
  \sphinxDUC{2500}{\sphinxunichar{2500}}
  \sphinxDUC{2502}{\sphinxunichar{2502}}
  \sphinxDUC{2514}{\sphinxunichar{2514}}
  \sphinxDUC{251C}{\sphinxunichar{251C}}
  \sphinxDUC{2572}{\textbackslash}
\fi
\usepackage{cmap}
\usepackage[T1]{fontenc}
\usepackage{amsmath,amssymb,amstext}
\usepackage{babel}



\usepackage{tgtermes}
\usepackage{tgheros}
\renewcommand{\ttdefault}{txtt}



\usepackage[Bjarne]{fncychap}
\usepackage{sphinx}

\fvset{fontsize=auto}
\usepackage{geometry}

\usepackage{alectryon}

% Include hyperref last.
\usepackage{hyperref}
% Fix anchor placement for figures with captions.
\usepackage{hypcap}% it must be loaded after hyperref.
% Set up styles of URL: it should be placed after hyperref.
\urlstyle{same}

\addto\captionsenglish{\renewcommand{\contentsname}{Contents:}}

\usepackage{sphinxmessages}
\setcounter{tocdepth}{1}



\title{alectryon\sphinxhyphen{}demo}
\date{ }
\release{}
\author{Clément Pit\sphinxhyphen{}Claudel}
\newcommand{\sphinxlogo}{\vbox{}}
\renewcommand{\releasename}{}
\makeindex
\begin{document}

\ifdefined\shorthandoff
  \ifnum\catcode`\=\string=\active\shorthandoff{=}\fi
  \ifnum\catcode`\"=\active\shorthandoff{"}\fi
\fi

\pagestyle{empty}
\sphinxmaketitle
\pagestyle{plain}
\sphinxtableofcontents
\pagestyle{normal}
\phantomsection\label{\detokenize{index::doc}}


\sphinxstepscope


\chapter{A chapter stored as a Coq file}
\label{\detokenize{coqchapter:a-chapter-stored-as-a-coq-file}}\label{\detokenize{coqchapter::doc}}
\begin{alectryon}
  % Generator: Alectryon
  \sep
  \begin{sentence}
    \begin{input}
      \PYG{k+kn}{Compute}~\PYG{o}{((}\PYG{k}{fun}~\PYG{o}{(}\PYG{+nv}{n}\PYG{o}{:}~nat\PYG{o}{)}~\PYG{o}{(}\PYG{+nv}{opt}\PYG{o}{:}~option~nat\PYG{o}{)}~\PYG{o}{(}\PYG{+nv}{eq}\PYG{o}{:}~opt~\PYG{o}{=}~Some~n\PYG{o}{)}~\PYG{o}{=\PYGZgt{}}~n\PYG{o}{)}\nl
      ~~~~~~~~~~~\PYGZus{}~\PYG{o}{(}Some~\PYG{l+m+mi}{3}\PYG{o}{)}~eq\PYGZus{}refl\PYG{o}{).}
    \end{input}
    \sep
    \begin{output}
      \begin{messages}
        \begin{message}
          \PYG{o}{=}~\PYG{l+m+mi}{3}\nl
          \PYG{o}{:}~nat
        \end{message}
      \end{messages}
    \end{output}
  \end{sentence}
\end{alectryon}

\sphinxstepscope


\chapter{MathJax in Sphinx}
\label{\detokenize{math:mathjax-in-sphinx}}\label{\detokenize{math::doc}}
\sphinxAtStartPar
Using any math on a page causes Sphinx to automatically load MathJax: \(e^{i\pi} = -1\).

\sphinxAtStartPar
If you want to highlight pieces of code with MathJax, too, then you can either:
\begin{itemize}
\item {} 
\sphinxAtStartPar
Use a custom mathjax config script (see the discussion in \sphinxcode{\sphinxupquote{recipes/mathjax.rst}}, the configuration in \sphinxcode{\sphinxupquote{conf.py}}, and the implementation in \sphinxcode{\sphinxupquote{recipes/sphinx/\_static/mathjax\_config.js}}).  Math in the following snippet is highlighted using this technique; look for the link to \sphinxcode{\sphinxupquote{mathjax\_config.js}} in the \sphinxcode{\sphinxupquote{\textless{}head\textgreater{}}} of this webpage:

\begin{alectryon}
  % Generator: Alectryon
  \sep
  \begin{sentence}
    \begin{input}
      \PYG{k+kn}{Notation}~\PYG{l+s+s2}{\PYGZdq{}\PYGZbs{}mathbb\PYGZob{}N\PYGZcb{}\PYGZdq{}}~\PYG{o}{:=}~nat\PYG{o}{.}\nl
    \end{input}
  \end{sentence}
  \sep
  \begin{sentence}
    \begin{input}
      \PYG{k+kn}{Print}~nat\PYG{o}{.}
    \end{input}
    \sep
    \begin{output}
      \begin{messages}
        \begin{message}
          \PYG{k+kn}{Inductive}~\PYG{+nf}{nat}~\PYG{o}{:}~\PYG{k+kt}{Set}~\PYG{o}{:=}\nl
          ~~~~O~\PYG{o}{:}~\PYG{o}{\PYGZbs{}}mathbb\PYG{o}{\PYGZob{}}N\PYG{o}{\PYGZcb{}}~\PYG{o}{|}~S~\PYG{o}{:}~\PYG{o}{\PYGZbs{}}mathbb\PYG{o}{\PYGZob{}}N\PYG{o}{\PYGZcb{}}~\PYG{o}{\PYGZhy{}\PYGZgt{}}~\PYG{o}{\PYGZbs{}}mathbb\PYG{o}{\PYGZob{}}N\PYG{o}{\PYGZcb{}.}\nl
          \nl
          \PYG{k+kn}{Arguments}~S~\PYGZus{}\PYG{o}{\PYGZpc{}}nat\PYGZus{}scope
        \end{message}
      \end{messages}
    \end{output}
  \end{sentence}
\end{alectryon}

\item {} 
\sphinxAtStartPar
Use a custom script to add class \sphinxcode{\sphinxupquote{mathjax\_process}} to each \sphinxcode{\sphinxupquote{.alectryon\sphinxhyphen{}io}} block, on a single page.  This works best if you need MathJax on just one page and you do not need to customize MathJax further.  Math in the following Coq snippet is highlighted that way (view the source of this page to see the custom script):



\begin{alectryon}
  % Generator: Alectryon
  \sep
  \begin{sentence}
    \begin{input}
      \PYG{k+kn}{Notation}~\PYG{l+s+s2}{\PYGZdq{}\PYGZbs{}mathbb\PYGZob{}B\PYGZcb{}\PYGZdq{}}~\PYG{o}{:=}~bool\PYG{o}{.}\nl
    \end{input}
  \end{sentence}
  \sep
  \begin{sentence}
    \begin{input}
      \PYG{k+kn}{Print}~bool\PYG{o}{.}
    \end{input}
    \sep
    \begin{output}
      \begin{messages}
        \begin{message}
          \PYG{k+kn}{Inductive}~\PYG{+nf}{bool}~\PYG{o}{:}~\PYG{k+kt}{Set}~\PYG{o}{:=}\nl
          ~~~~true~\PYG{o}{:}~\PYG{o}{\PYGZbs{}}mathbb\PYG{o}{\PYGZob{}}B\PYG{o}{\PYGZcb{}}~\PYG{o}{|}~false~\PYG{o}{:}~\PYG{o}{\PYGZbs{}}mathbb\PYG{o}{\PYGZob{}}B\PYG{o}{\PYGZcb{}.}
        \end{message}
      \end{messages}
    \end{output}
  \end{sentence}
\end{alectryon}

\end{itemize}

\sphinxstepscope


\chapter{Integration with MyST}
\label{\detokenize{MyST:integration-with-myst}}\label{\detokenize{MyST::doc}}
\sphinxAtStartPar
To combine Alectryon and MyST (a Markdown parser with support for docutils/Sphinx directives), just load both plugins in your Sphinx configuration:

\begin{sphinxVerbatim}[commandchars=\\\{\}]
extensions = [\PYGZdq{}alectryon.sphinx\PYGZdq{}, \PYGZdq{}myst\PYGZus{}parser\PYGZdq{}]
\end{sphinxVerbatim}

\sphinxAtStartPar
That’s enough to run Coq fragments and link to identifiers:

\begin{alectryon}
  % Generator: Alectryon
  \sep
  \begin{sentence}
    \begin{input}
      \PYG{k+kn}{Print}~nat\PYG{o}{.}
    \end{input}
    \sep
    \begin{output}
      \begin{messages}
        \begin{message}
          \PYG{k+kn}{Inductive}~\PYG{+nf}{nat}~\PYG{o}{:}~\PYG{k+kt}{Set}~\PYG{o}{:=}~~O~\PYG{o}{:}~nat~\PYG{o}{|}~S~\PYG{o}{:}~nat~\PYG{o}{\PYGZhy{}\PYGZgt{}}~nat\PYG{o}{.}\nl
          \nl
          \PYG{k+kn}{Arguments}~S~\PYGZus{}\PYG{o}{\PYGZpc{}}nat\PYGZus{}scope
        \end{message}
      \end{messages}
    \end{output}
  \end{sentence}
\end{alectryon}

\sphinxAtStartPar
For roles use \sphinxcode{\sphinxupquote{\{role\}\textasciigrave{}argument\textasciigrave{}}} syntax: \sphinxstyleemphasis{\sphinxhref{https://coq.inria.fr/library/Coq.Init.Nat.html\#even}{like this}}.  For math use either the \sphinxcode{\sphinxupquote{\{math\}}} role (\(e^{i\pi} = -1\)) or \sphinxcode{\sphinxupquote{\$}} signs (with the \sphinxcode{\sphinxupquote{dollarmath}} extension, see \sphinxcode{\sphinxupquote{conf.py}}): (\(\cos(\pi) = -1\)).

\sphinxAtStartPar
Note that MyST disables MathJax’s heuristics for finding text to process (by marking the root of the document with \sphinxcode{\sphinxupquote{mathjax\_ignore}}), so any math outside of \sphinxcode{\sphinxupquote{\{math\}}} or \sphinxcode{\sphinxupquote{\$}} delimiters is not processed: \textbackslash{}(this is not math\textbackslash{}); use the \sphinxcode{\sphinxupquote{mathjax\_process}} HTML class to revert that.

\sphinxAtStartPar
The default role produces Coq code: \sphinxcode{\sphinxupquote{\DUrole{k}{let} \DUrole{n}{a} \DUrole{o}{:=} \DUrole{mi}{1} \DUrole{k}{in} \DUrole{n}{a} \DUrole{o}{+} \DUrole{n}{a}}}.

\begin{alectryon}
  % Generator: Alectryon
  \sep
  \begin{sentence}
    \begin{input}
      \PYG{k+kn}{Definition}~\PYG{+nf}{example\PYGZus{}from\PYGZus{}sphinx}\PYG{o}{:}~nat\PYG{o}{.}
    \end{input}
    \sep
    \begin{output}
      \begin{goals}
        \begin{goal}
          \begin{hyps}\end{hyps}
          \sep
          \infrule{}
          \sep
          \begin{conclusion}
            nat
          \end{conclusion}
        \end{goal}
      \end{goals}
    \end{output}
  \end{sentence}
  \sep
  \begin{sentence}
    \begin{input}
      \PYG{k+kn}{Proof}\PYG{o}{.}\nl
    \end{input}
  \end{sentence}
  \sep
  \begin{sentence}
    \begin{input}
      ~~\PYG{+nb}{simple~apply}~Nat\PYG{o}{.}add\PYG{o}{.}
    \end{input}
    \sep
    \begin{output}
      \begin{goals}
        \begin{goal}
          \begin{hyps}\end{hyps}
          \sep
          \infrule{}
          \sep
          \begin{conclusion}
            nat
          \end{conclusion}
        \end{goal}
        \sep
        \begin{extragoals}
          \begin{goal}
            \begin{hyps}\end{hyps}
            \sep
            \infrule{}
            \sep
            \begin{conclusion}
              nat
            \end{conclusion}
          \end{goal}
        \end{extragoals}
      \end{goals}
    \end{output}
  \end{sentence}
  \sep
  \begin{sentence}
    \begin{input}
      ~~\PYG{+nb+bp}{exact}~\PYG{l+m+mi}{1}\PYG{o}{.}
    \end{input}
    \sep
    \begin{output}
      \begin{goals}
        \begin{goal}
          \begin{hyps}\end{hyps}
          \sep
          \infrule{}
          \sep
          \begin{conclusion}
            nat
          \end{conclusion}
        \end{goal}
      \end{goals}
    \end{output}
  \end{sentence}
  \sep
  \begin{sentence}
    \begin{input}
      ~~\PYG{+nb}{assert}~\PYG{o}{(}n\PYG{o}{:}~nat\PYG{o}{).}
    \end{input}
    \sep
    \begin{output}
      \begin{goals}
        \begin{goal}
          \begin{hyps}\end{hyps}
          \sep
          \infrule{}
          \sep
          \begin{conclusion}
            nat
          \end{conclusion}
        \end{goal}
        \sep
        \begin{extragoals}
          \begin{goal}
            \begin{hyps}
              \hyp{n}{nat}
            \end{hyps}
            \sep
            \infrule{}
            \sep
            \begin{conclusion}
              nat
            \end{conclusion}
          \end{goal}
        \end{extragoals}
      \end{goals}
    \end{output}
  \end{sentence}
  \sep
  \begin{sentence}
    \begin{input}
      ~~\PYG{l+m+mi}{2}\PYG{o}{:}~\PYG{+nb}{clear}~n\PYG{o}{.}
    \end{input}
    \sep
    \begin{output}
      \begin{goals}
        \begin{goal}
          \begin{hyps}\end{hyps}
          \sep
          \infrule{}
          \sep
          \begin{conclusion}
            nat
          \end{conclusion}
        \end{goal}
        \sep
        \begin{extragoals}
          \begin{goal}
            \begin{hyps}\end{hyps}
            \sep
            \infrule{}
            \sep
            \begin{conclusion}
              nat
            \end{conclusion}
          \end{goal}
        \end{extragoals}
      \end{goals}
    \end{output}
  \end{sentence}
  \sep
  \begin{sentence}
    \begin{input}
      ~~\PYG{+nb+bp}{exact}~\PYG{l+m+mi}{3}\PYG{o}{.}
    \end{input}
    \sep
    \begin{output}
      \begin{goals}
        \begin{goal}
          \begin{hyps}\end{hyps}
          \sep
          \infrule{}
          \sep
          \begin{conclusion}
            nat
          \end{conclusion}
        \end{goal}
      \end{goals}
    \end{output}
  \end{sentence}
  \sep
  \begin{sentence}
    \begin{input}
      ~~\PYG{+nb+bp}{exact}~\PYG{l+m+mi}{2}\PYG{o}{.}\nl
    \end{input}
  \end{sentence}
  \sep
  \begin{sentence}
    \begin{input}
      \PYG{k+kn}{Defined}\PYG{o}{.}\nl
    \end{input}
  \end{sentence}
  \sep
  \begin{txt}
    \nl
  \end{txt}
  \sep
  \begin{sentence}
    \begin{input}
      \PYG{k+kn}{Print}~example\PYGZus{}from\PYGZus{}sphinx\PYG{o}{.}
    \end{input}
    \sep
    \begin{output}
      \begin{messages}
        \begin{message}
          example\PYGZus{}from\PYGZus{}sphinx~\PYG{o}{=}~\PYG{l+m+mi}{1}~\PYG{o}{+}~\PYG{o}{(}\PYG{k}{let}~\PYG{+nv}{n}~\PYG{o}{:=}~\PYG{l+m+mi}{3}~\PYG{k}{in}~\PYG{l+m+mi}{2}\PYG{o}{)}\nl
          ~~~~~\PYG{o}{:}~nat
        \end{message}
      \end{messages}
    \end{output}
  \end{sentence}
\end{alectryon}


\chapter{Indices and tables}
\label{\detokenize{index:indices-and-tables}}\begin{itemize}
\item {} 
\sphinxAtStartPar
\DUrole{xref,std,std-ref}{genindex}

\item {} 
\sphinxAtStartPar
\DUrole{xref,std,std-ref}{modindex}

\item {} 
\sphinxAtStartPar
\DUrole{xref,std,std-ref}{search}

\end{itemize}



\renewcommand{\indexname}{Index}
\printindex
\end{document}